% regexapplicative-Rr.tex
\begin{hcarentry}[new]{regex-applicative}
\report{Roman Cheplyaka}%11/11
\status{active development}
\makeheader

regex-applicative is aimed to be an efficient and easy to use parsing combinator
library for Haskell based on regular expressions.

There are several ways in which one can specify what part of the string should
be matched: the whole string, a prefix or an arbitrary part ("leftmost infix")
of the string.

Additionally, for prefix and infix modes, one can demand either
the longest part, the shortest part or the first (in the left-biased ordering)
part.

Finally, other things being equal, submatches are chosen using left bias.

Recently the performance has been improved by using more efficient algorithm for
the parts of the regular expression whose result is not used.

\noindent
Example code can be found on the
\href{https://github.com/feuerbach/regex-applicative/wiki/Examples}
{wiki}.

\FurtherReading
\begin{compactitem}
  \item \url{http://hackage.haskell.org/package/regex-applicative}
  \item \url{http://github.com/feuerbach/regex-applicative}
\end{compactitem}
\end{hcarentry}

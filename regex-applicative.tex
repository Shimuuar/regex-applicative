% regexapplicative-Rr.tex
\begin{hcarentry}[new]{regex-applicative}
\report{Roman Cheplyaka}%11/11
\status{active development}
\makeheader

regex-applicative is aimed to be an efficient and easy to use parsing combinator library for Haskell based on regular expressions.

Regular expressions have Perl-like (left-biased) semantics to satisfy most of
the daily regex needs, but also allow longest matching prefix search
useful for
lexical analysis.

For example, the following code finds filename extensions:

\begin{verbatim}
import Text.Regex.Applicative

getExtension :: String -> Maybe String
getExtension str =
    str =~
        many anySym *>
        sym '.'     *>
        many anySym
\end{verbatim}

\noindent
More examples can be found on the
\href{https://github.com/feuerbach/regex-applicative/wiki/Examples}
{wiki}.

\FurtherReading
\begin{compactitem}
  \item \url{http://hackage.haskell.org/package/regex-applicative}
  \item \url{http://github.com/feuerbach/regex-applicative}
\end{compactitem}
\end{hcarentry}
